\subsection{Fast multipole method}

The fast multipole method (\fmm) is an algorithm that can reduce the quadratic time and space complexity of such matrix-vector multiplication down to $\mathcal{O}(N)$.
In the context of \fmm, $\{x_i\}$ and $\{y_j\}$ in equation ? are often referred to as the set of targets and sources respectively, with $\{q_j\}$ representing the source densities (charges).
The goal of \fmm is to efficiently compute the potential at $N$ targets $\{s_i\}$ induced by all $N$ sources and the kernel function $g$.

\fmm relies on two fundamental ideas: (1) approximating the far-range interactions between distant clusters of sources and targets using low-rank methods, while computing the near-range interactions exactly, and (2) partitioning the domain using a tree structure to maximize the far-range portion in the computation.
The first step of the algorithm is to recursively subdivide the domain until each leaf only has a constant number of bodies inside.
Figure shows a 3-level quadtree, where the yellow region is in the near-field of node B, and the blue region is in the far-field.