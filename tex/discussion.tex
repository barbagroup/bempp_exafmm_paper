%!TEX root = main.tex
With this paper, we introduce a new platform for computational investigations in biomolecular electrostatics, combining high-performance with high researcher productivity. 
Bempp-Exafmm integrates one of the most trusted boundary element software packages with one of the most performant fast-summation libraries using multipole methods. 
A Python entry point gives researchers ease of use, while enabling computational research at virus scale on standard workstations.
The software is open source under permissive public licenses, and developed in the open.

We present several results that confirm the usefulness of the platform, verify solution correctness with classic benchmarks, and showcase the performance. 
% conditioning
In section \ref{result_conditioning}, we compared the conditioning of the interior and exterior derivative formulations.
Despite the fact that both yield a well-conditioned system, where the condition number does not grow with the problem size, the exterior formulation always converges faster due to the clustering of its eigenvalues.
It shows a greater advantage over the interior formulation as the difference between $\epsilon_1$ and $\epsilon_2$ becomes larger.
Previous publications have used one or the other of these solution methods, but we found no comparison of the two in the literature.
Our experiments showed that the exterior formulation converges twice as fast with typical values of the permittivities: an important advantage.
Seeing this, we quickly computed condition numbers, made heatmaps of the matrices, and plotted eigenvalues on the complex plane---all interactively, in a Jupyter notebook---leading to an explanation of the different algebraic behaviors.
(Readers can find a final Jupyter notebook in the GitHub repository for this manuscript.)
This study also serves as example of how users can benefit from our high-productivity platform.
Through interactive computing, users can adapt various formulations, try out different problem setups, analyze intermediate results on-the-fly without the hassle of recompilation.

% mesh refinement
We performed two mesh-refinement studies to verify Bempp-Exafmm, using a spherical molecule with an off-center charge, and using a real biomolecule (bovine pancreatic trypsin inhibitor).
In the former study, we compared with the analytical solution; in the latter, we compared with an approximate value from Richardson extrapolation.
In both cases, we used the direct formulation and the exterior derivative formulation.
To reveal the discretization error, we set the \fmm expansion order to 10 to achieve 9 digits of accuracy.
The error of the computed solvation energy decays linearly with respect to $N$, as shown in Figure \ref{fig:sphere_convergence} and \ref{fig:5PTI_convergence}.
These results provide evidence of the code correctness for computing biomolecular solvation energies.

% performance and Zika virus
In section \ref{result_performance}, we elaborate on the performance of Bempp-Exafmm for different problem sizes using both formulations.
The linear complexity of the assembly time (Figure \ref{fig:sphere_assembly_time}) and \fmm time (Figure \ref{fig:sphere_fmm}) guarantees the overall linear time complexity of Bempp-Exafmm, which, together with the linear space complexity shown in Figure \ref{fig:sphere_memory}, makes it feasible to perform large-scale simulations on a workstation.
Table \ref{tab:sphere_time} lists the timings in detail.
Despite being ill-conditioned, the direct formulation still shows an advantage in terms of the overall time for smaller problem sizes.
Conversely, the derivative formulation shines in larger problems.
The crossover point should be problem- and hardware-specific.
Finally, we computed the solvation energy of a Zika virus using both formulations and verified our results against \pygbe in \ref{result_zika}.
The linear system, for a mesh containing 10 million boundary elements, was solved in 80 minutes on a single node.
It shows that the performance of our code is in the same ballpark as other state-of-the-art fast BEM PB solvers \cite{GengKrasny2013,ZhangPengHuangPitsianisSunLu2015,CooperBardhanBarba2014} and gives us confidence in its capability of solving virus-scale problems.

Poisson-Boltzmann solvers have been around for decades, available as stand-alone applications and web servers, and using a variety of solution approaches \cite{JurrusETal2018} ranging from finite difference, to finite element, and boundary element methods.
Some solvers are integrated into a number of computational workflows that use them for mean field potential visualization \cite{HumphreyETal1996} and free energy calculations \cite{MillerETal2012,KumariETal2014}, usually interfaced through bash or Python scripts.
Our approach also becomes useful as part of existing computational workflows, such as MMPBSA methods \cite{WangETal2018}.
The Poisson-Boltzmann model has known limitations, being the rigid approximation of the solute a specially difficult one at large scales. 
Regardless, the electrostatic potential and polar solvation free energy are valuable information for analysis of, for example, the size and structure of a virus \cite{SiberETal2012}, or its pH dependence \cite{RoshalETal2019}. 

The modern design of Bempp is built such that high-performance computations are accessible from a high-productivity language.
This makes our effort stand out in the current landscape of Poisson-Boltzmann solvers in three ways: interoperability, ease of use, and robustness. 
\begin{enumerate}
\item Interoperability: Bempp is written in Python, and hence, is callable from a Jupyter notebook. This fits naturally in any computational workflow that uses Jupyter notebooks, for example, with openMM \cite{EastmanETal2017}, Biobb \cite{AndrioETal2019}, MDAnalysis \cite{GowersETal2019}, pytraj \cite{RoeCheatham2013}, or PyMOL \cite{PyMOL}. The Jupyter Notebook becomes a computational glue across models and scales; no interface script required. 

\item Ease of use: Python and Jupyter notebooks are widely used, even in non-computational settings. Bempp is easily installed through \texttt{conda}, and gives a result in less than 20 lines of code. This, moreover, using parallel and state-of-the-art algorithms in a way that is almost transparent to the user, allowing for large-scale simulations on workstations or small clusters.
A thin layer separates the application and Bempp, giving a more experienced user access to develop new models, for example, through the \fmm-\bem coupling capability of Bempp.

\item Robustness: Bempp is actively developed with high standards of software engineering, such as unit and system testing, continuous integration, etc. It was originally designed for scattering problems, impacting a large group of people, well beyond the molecular simulation community. This builds high trust and reliability of the code, as it is thoroughly tested in a diverse set of applications. The software has a better chance to survive in the long term, and any improvements done by people in other domains will have an effect on its use to solve the Poisson-Boltzmann equation. 

\end{enumerate}

Many popular molecular simulation software packages exist, designed for different applications, scales, quantities of interest, etc.
This has led to community-wide efforts, such as BioExcel (\url{https://bioexcel.eu/}) and MolSSI (\url{http://molssi.org}), that are looking for a common ground between them, as well as promoting good software development practices for robust and easy-to-use codes.
This standard is very much aligned with our work.

Modern research software efforts today aim for the union of high performance and high researcher productivity.
A vigorous trend is unmistakable towards empowering users with interactive computing, particularly using Jupyter notebooks. 
Our work contributes a platform for interactive investigations in biomolecular electrostatics that is easy to use, easy to install, highly performant, extensible, open source and free.
We contemplate a bright future for science domains that gel community efforts to jointly develop and curate software tools with similar philosophy.
