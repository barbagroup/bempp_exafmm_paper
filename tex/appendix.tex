\small{

% add comparison with results from APBS
\section{Comparison with trusted community software} \label{comparison}
We build more confidence on Bempp-Exafmm by comparing our results with a well-established finite-difference code APBS (version 1.5), using 5PTI and 8 other molecules.
We prepared the molecular structures using the \texttt{charmm} force field.
For each molecule, we created three successive meshes to study the convergence with each software.
For Bempp-Exafmm, we used \texttt{Nanoshaper} to generate three surface meshes with the element density set to 4, 8, 16 ${\si{\angstrom}}^{-2}$, respectively.
In APBS, we controlled the multigrid parameters to ensure that three grid spacings are refined with a constant factor of 2.
All mesh sizes are listed in Appendix \ref{sec:apbs_mesh}.
Other parameters are the same as our previous grid-convergence study, as listed in Table \ref{tab:convergence}.

Table \ref{tab:APBS_result} presents the convergence results of both codes using a set of molecules.
Bempp used the derivative formulation throughout.
In each case, we obtained an extrapolated value of the solvation energy of each molecule, which serves as the reference value to estimate the discretization error with each software.
As expected, the solvation energy computed from Bempp converges at the rate of $\mathcal{O}(N^{-1})$, where $N$ is the number of elements.
With APBS, the average observed order of convergence over all cases is 1.25, with respect to the grid spacing $h$; other studies \cite{CooperBardhanBarba2014,GengKrasny2013} have reported a similar rate. 
Our $\mathcal{O}(N^{-1})$ convergence corresponds to $\mathcal{O}(h^{-2})$ convergence in grid size, but for boundary elements (surface triangles) it is more natural to look at convergence with respect to $N$.
The difference in the extrapolated solvation energies obtained with the two codes varies between 0.8\% and 1.8\% across all cases.
A discrepancy is expected because Bempp-Exafmm and APBS use different geometrical representations for the molecular surface, while the atomic charges in the molecule are mapped to grid points through interpolation in APBS.
(We also compared our results with MIBPB \cite{chen2011mibpb} using only the finest mesh: the details can be found in Appendix \ref{sec:comp_mibpb}.)

\begin{table*}[]
    \centering
    \resizebox{\textwidth}{!}{%
    \begin{tabular}{cc|ccc|cc|ccc|cc}
      &   & \multicolumn{5}{c|}{APBS}                                                                    & \multicolumn{5}{c}{Bempp}                                                                   \\
      &   & \multicolumn{3}{c|}{Error}     & \multirow{2}{*}{$\Delta G_{solv}$} & \multirow{2}{*}{Order$(1/h)$} & \multicolumn{3}{c|}{Error}     & \multirow{2}{*}{$\Delta G_{solv}$} & \multirow{2}{*}{Order$(1/N)$} \\
    ID   & $N_{atoms}$ & coarse   & medium   & fine     &                                    &                        & coarse   & medium   & fine     &                                    &                        \\ \hline
    1AJJ & 513 & 4.40e-02 & 1.33e-02 & 4.00e-03 & -266.15                            & 1.73                   & 2.75e-02 & 1.12e-02 & 4.59e-03 & -268.67                            & 1.29                   \\
    1VJW & 826 & 3.09e-02 & 1.45e-02 & 6.77e-03 & -297.06                            & 1.09                   & 4.92e-02 & 2.23e-02 & 1.01e-02 & -302.50                            & 1.14                   \\
    5PTI & 892 & 3.68e-02 & 1.38e-02 & 5.14e-03 & -311.69                            & 1.42                   & 5.12e-02 & 2.23e-02 & 9.67e-03 & -314.34                            & 1.20                   \\
    1R69 & 997 & 4.31e-02 & 2.10e-02 & 1.02e-02 & -261.02                            & 1.04                   & 5.09e-02 & 2.34e-02 & 1.08e-02 & -265.02                            & 1.12                   \\
    1A2S & 1272 & 5.25e-02 & 2.40e-02 & 1.10e-02 & -456.56                            & 1.13                   & 4.21e-02 & 1.93e-02 & 8.86e-03 & -461.25                            & 1.12                   \\
    1SVR & 1433 & 5.28e-02 & 2.21e-02 & 9.28e-03 & -393.45                            & 1.25                   & 6.48e-02 & 2.89e-02 & 1.29e-02 & -398.83                            & 1.16                   \\
    1A63 & 2065 & 4.51e-02 & 1.88e-02 & 7.82e-03 & -559.39                            & 1.26                   & 5.82e-02 & 2.60e-02 & 1.16e-02 & -567.24                            & 1.16                   \\
    1A7M & 2804 & 4.13e-02 & 1.88e-02 & 8.53e-03 & -524.29                            & 1.14                   & 4.93e-02 & 2.24e-02 & 1.02e-02 & -531.48                            & 1.14                   \\
    1F6W & 8247 & 6.45e-02 & 2.78e-02 & 1.20e-02 & -1277.51                           & 1.21                   & 4.18e-02 & 1.80e-02 & 7.76e-03 & -1301.08                           & 1.22                     
    \end{tabular}
    }
    \caption{Convergence results of the solvation energy of 9 molecules using APBS and Bempp with derivative formulation.
    The error is with respect to an extrapolated value of the solvation energy using Richardson extrapolation.
    The solvation energy $\Delta G_{solv}$ is in units of kcal/mol.
    The observed order of convergence is with respect to the grid spacing $h$ for APBS (volumetric-based solver) and with respect to the number of elements $N$ for Bempp (boundary-element solver).}
    \label{tab:APBS_result}
\end{table*}

\section{Meshes used in the convergence study of APBS and Bempp}\label{sec:apbs_mesh}
Table \ref{tab:APBS_mesh} summarizes the mesh sizes of the molecules used in the convergence study of APBS and Bempp.
The former uses Cartesian grids while the latter uses surface meshes generated by \texttt{Nanoshaper}.

\begin{table}[]
    \centering
    \resizebox{\columnwidth}{!}{%
    \begin{tabular}{c|ccc|ccc}
    \multirow{2}{*}{ID} & \multicolumn{3}{c|}{APBS} & \multicolumn{3}{c}{Bempp} \\
                        & coarse  & medium  & fine & coarse  & medium  & fine   \\ \hline
    1AJJ                & $65 ^3$     & $129^3$     & $257^3$  & 8544    & 17200   & 34468  \\
    1VJW                & $97 ^3$     & $193^3$     & $385^3$  & 11544   & 23296   & 46660  \\
    5PTI                & $97 ^3$     & $193^3$     & $385^3$  & 12512   & 25204   & 50596  \\
    1R69                & $97 ^3$     & $193^3$     & $385^3$  & 12004   & 24216   & 48648  \\
    1A2S                & $65 ^3$     & $129^3$     & $257^3$  & 17784   & 35732   & 71648  \\
    1SVR                & $97 ^3$     & $193^3$     & $385^3$  & 18644   & 37348   & 75004  \\
    1A63                & $129^3$     & $257^3$     & $513^3$  & 27996   & 56356   & 113124 \\
    1A7M                & $129^3$     & $257^3$     & $513^3$  & 30576   & 61652   & 123776 \\
    1F6W                & $129^3$     & $257^3$     & $513^3$  & 77464   & 156140  & 313692
    \end{tabular}
    }
    \caption{Mesh sizes of the molecules used in the convergence study of APBS and Bempp.}
    \label{tab:APBS_mesh}
\end{table}

\section{Comparison with MIBPB}\label{sec:comp_mibpb}
We also compared the solvation energy computed from Bempp and another trusted finite-difference code MIBPB using the same set of molecules.
The version we used was downloaded from \href{https://weilab.math.msu.edu/MIBPB}{https://weilab.math.msu.edu/MIBPB}.
We enabled the linearized PB solver option in MIBPB and set the grid resolution parameter to 0.25 to produce finer grids.
Correspondingly, we selected Bempp's results from the finest mesh, whose size is shown in Table \ref{tab:APBS_mesh}, to compare.
Other parameters are the same as those in the grid-convergence study, listed in Table \ref{tab:convergence}.
Table \ref{tab:MIBPB_result} presents the grid sizes used in MIBPB and the solvation energy computed from both software.
The difference in $\Delta G_{solv}$ varies between 0.05\% and 1.9\% across all molecules.

\begin{table}[]
    \centering
    \resizebox{0.8\columnwidth}{!}{%
    \begin{tabular}{c|c|c|c}
    \multirow{2}{*}{ID} & \multicolumn{2}{c|}{MIBPB}      & Bempp             \\
                        & grid size   & $\Delta G_{solv}$ & $\Delta G_{solv}$ \\ \hline
    1AJJ                & $132\times 142\times 154$ & -271.59           & -269.90           \\
    1VJW                & $137\times 160\times 148$ & -307.71           & -305.56           \\
    5PTI                & $145\times 158\times 178$ & -317.54           & -317.37           \\
    1R69                & $167\times 155\times 148$ & -270.94           & -267.88           \\
    1A2S                & $173\times 166\times 186$ & -463.11           & -465.34           \\
    1SVR                & $174\times 187\times 192$ & -410.40           & -403.97           \\
    1A63                & $283\times 184\times 186$ & -581.02           & -573.83           \\
    1A7M                & $191\times 266\times 225$ & -531.93           & -536.91           \\
    1F6W                & $330\times 272\times 309$ & -1336.48          & -1311.17         
    \end{tabular}
    }
    \caption{MIBPB and Bempp results of the solvation energy $\Delta G_{solv}$ of different molecules, in units of kcal/mol.}
    \label{tab:MIBPB_result}
\end{table}


}
