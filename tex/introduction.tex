%!TEX root = main.tex
Electrostatics plays a key role in the structure and function of biological molecules.
Long-range electrostatic effects intervene in various essential processes, such as protein binding, with biomolecules always present in a solution of water with ions.
Computer simulations to study electrostatic interactions in biomolecular systems divide into those that represent the solvent explicitly---in full atomic detail---or implicitly.
In so-called implicit-solvent models~\cite{RouxSimonson1999,DecherchiETal2015}, the solvent degrees of freedom are averaged out in a continuum description.
Starting from electrostatic theory, this leads to a mathematical model based on the Poisson-Boltzmann equation, and widely used to compute mean-field electrostatic potentials and solvation free energies.
Poisson-Boltzmann solvers have been numerically implemented using finite difference \cite{RocchiaAlexovHonig2001, BakerETal2001}, finite element \cite{BakerETal2001,BondETal2010,HolstETal2012}, boundary element \cite{AltmanBardhanWhiteTidor2009, GengKrasny2013, ZhangPengHuangPitsianisSunLu2015, CooperBardhanBarba2014}, and (semi) analytical \cite{LotanHead-Gordon2006,FelbergETal2017} methods, scaling up to problems as large as virus capsids \cite{ZhangETal2019,MartinezETal2019}.

Virus-scale simulations are at the limit of what can be accomplished in computational biophysics, using leadership computing facilities.
The first explicit-solvent atomic simulation of a virus using molecular dynamics was published just 15 years ago, modeling a plant virus (satellite tobacco mosaic virus) of 1.7 nm in diameter \cite{FreddolinoETal2006}.
The full model included 1 million atoms, and the computations ran for many days on the world-class facilities at the National Center for Supercomputing Application (NCSA), University of Illinois.
Using largely the same methods, researchers just last year could model the full viral envelope of a 2009 pandemic influenza A H1N1 virus, with a diameter of about 115 nm \cite{DurrantETal2020}.
In this case, the full system consisted of 160 million atoms, and the computations ran on the Blue Waters supercomputer at NCSA using 115k processor cores (4,096 physical nodes).
This is among the largest biomolecular systems ever simulated using all-atom molecular dynamics.

Only a few elite researchers can access these leadership computing facilities, however, and if molecular science of viruses is to progress, computational tools that are more widely accessible are needed.
The vision behind this paper is to build an electrostatic simulation platform for biomolecular applications that allows researchers to access it via the Python/Jupyter ecosystem. This provides a high degree of flexibility in the underlying formulations, rapid prototyping of novel models, ease of deployment and integration into existing simulation processes.

To achieve this vision, we are coupling two libraries, the high-level Galerkin boundary element library Bempp, which is fully developed in Python, and the very fast low-level high-performance fast multipole method (FMM) library Exafmm. 
Boundary integral problems are described in Bempp using a high-level approach that allows to build up even complex block-operator systems in just a few lines of code. Bempp then executes the discretization, depending on the chosen parameters and machine environment. In the case of FMM acceleration, Exafmm is called essentially as a matrix-vector black-box below the user level, hiding any technicalities associated with the discretization.

This approach has the following advantages as compared to an integrated PB solver in, for example, C++:
\begin{itemize}
	\item \textit{Strict Separation of Concerns}. The user-level description of the electrostatic problem is completely separated from the underlying discretization routines and the FMM coupling. One can easily move between different types of implementations (e.g. dense discretization, FMM) with a single parameter change, change input file handling or postprocessing.
	\item \textit{Fast Prototyping of different formulations}. We present in this paper results produced with a direct formulation and a derivative Juffer type formulations. To implement these different formulations is just a small changes in a few lines of high-level code. The user can easily experiment with other models, such as piecewise solvation models with different solvation parameters in each layer.
	\item \textit{Portability}. Bempp and Exafmm can easily be installed as a joint Docker image that is automatically tracking the current development of these libraries. The whole solver code can be implemented in a small Jupyter notebook.
\end{itemize}
A high-level productive approach does not fully come without costs. A dedicated highly specialized C++ code that integrates all steps tends to be faster than our solution. Nevertheless, in this paper we demonstrate that our code is highly competitive for real-world solvation energy computations (and many other electrostatic computations), while preserving full flexibility through the use of a high-productivity Python environment.



