Add introduction here.


The motivation of our test case is modeling molecular solvation using an implicit-solvent model, where the solvent degrees of freedom are averaged out in a continuum description.
Using electrostatic theory, we can describe it with a PDE-based model, where the Poisson-Boltzmann equation governs in the solvent, and the Poisson equation in the solute molecule.
Poisson-Boltzmann solvers are widely used in the biophysics and biochemistry communities to compute mean-field electrostatic potentials and solvation free energies, and they have been implemented using finite difference \cite{delphi, apbs}, finite element \cite{apbs}, boundary element \cite{afmpb, tabi, pygbe}, and (semi) analytical \cite{thg} models. 
In particular, using a boundary element method has proven to be very effective to tackle large-scale problems, such as virus capsids, on HPC infrastructure \cite{lu}.
In our case, however, our main focus is to keep an efficient, yet simple, code, that can be accessed from a Jupyter notebook.
