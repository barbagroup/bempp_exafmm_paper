%!TEX root = main.tex
Electrostatics plays a key role in the structure and function of biological molecules.
Long-range electrostatic effects intervene in various essential processes, such as protein binding, with biomolecules always present in a solution of water with ions.
Computer simulations to study electrostatic interactions in biomolecular systems divide into those that represent the solvent explicitly---in full atomic detail---or implicitly.
In so-called implicit-solvent models~\cite{RouxSimonson1999,DecherchiETal2015}, the solvent degrees of freedom are averaged out in a continuum description.
Starting from electrostatic theory, this leads to a mathematical model based on the linearized Poisson-Boltzmann equation, and widely used to compute mean-field electrostatic potentials and solvation free energies.
Poisson-Boltzmann solvers have been numerically implemented using finite difference \cite{RocchiaAlexovHonig2001, BakerETal2001}, finite element \cite{BakerETal2001,BondETal2010,HolstETal2012}, boundary element \cite{AltmanBardhanWhiteTidor2009, GengKrasny2013, ZhangPengHuangPitsianisSunLu2015, CooperBardhanBarba2014}, and (semi) analytical \cite{LotanHead-Gordon2006,FelbergETal2017} methods, scaling up to problems as large as virus capsids \cite{ZhangETal2019,MartinezETal2019}.

Virus-scale simulations are at the limit of what can be accomplished in computational biophysics, using leadership computing facilities.
These computational studies provide fundamental insight to understand the physical underpinnings of viruses \cite{HaddenPerilla2018}, such as their mechanical stability \cite{ArkhipovETal2009}, binding mechanisms \cite{DurrantETal2020}, assembly \cite{DickETal2018}, among others. 
The first explicit-solvent atomic simulation of a virus using molecular dynamics was published just 15 years ago, modeling a plant virus (satellite tobacco mosaic virus) of 1.7 nm in diameter \cite{FreddolinoETal2006}.
The full model included 1 million atoms, and the computations ran for many days on the world-class facilities at the National Center for Supercomputing Application (NCSA), University of Illinois.
Using largely the same methods, researchers just last year could model the full viral envelope of a 2009 pandemic influenza A H1N1 virus, with a diameter of about 115 nm \cite{DurrantETal2020}.
In this case, the full system consisted of 160 million atoms, and the computations ran on the Blue Waters supercomputer at NCSA using 115k processor cores (4,096 physical nodes).
This is among the largest biomolecular systems ever simulated using all-atom molecular dynamics.

Only a few elite researchers can access these leadership computing facilities, however, and if molecular science of viruses is to progress, computational tools that are more widely accessible are needed.
It is in this context where approximate coarse grained \cite{ReddySansom2016,MachadoETal2017,HuberETal2021} and continuum models \cite{MartinezETal2019} play a key role.
The vision behind this paper is to build an electrostatic simulation platform for biomolecular applications that allows researchers to access it via the Python/Jupyter ecosystem. This provides a high degree of flexibility in the underlying formulations, rapid prototyping of novel models, ease of deployment and integration into existing simulation workflows.

To achieve this vision, we are coupling two libraries, the high-level Galerkin boundary element library Bempp, which is fully developed in Python, and the very fast low-level high-performance fast multipole method (\fmm) library Exafmm. 
Boundary integral problems are described in Bempp using a high-level approach that enables building even complex block-operator systems in just a few lines of Python code. Bempp then executes the discretization, depending on the chosen parameters and machine environment. 
Exafmm is called as a matrix-vector black-box below the user level, hiding all technicalities associated with the discretization.

This approach has the following advantages as compared to an integrated PB solver implemented in, for example, C++:
\begin{itemize}
	\item \textit{Strict separation of concerns}. The user-level description of the electrostatic problem is completely separated from the underlying discretization routines and the \fmm coupling. One can easily move between different types of implementations (e.g., dense discretization, \fmm) editing a single parameter, change input file handling or postprocessing.
	\item \textit{Fast prototyping of different formulations}. We present in this paper results produced with a direct formulation and derivative or Juffer-type formulations. Applying these different formulations requires editing just a few lines of high-level code. The user can easily experiment with other models, such as piecewise solvation models with different solvation parameters in each layer.
	\item \textit{Portability}. Bempp and Exafmm can easily be installed as a joint Docker image that is automatically tracking the current development of these libraries. The whole solution workflow can be implemented in a brief Jupyter notebook.
\end{itemize}
A high-level productive approach does come with some costs. A dedicated highly specialized C++ code that integrates all steps might be faster than our solution. Nevertheless, in this paper we demonstrate that our software platform is highly competitive for real-world solvation energy computations (and many other electrostatic computations), while preserving full flexibility through the use of a high-productivity Python environment.

We present results that show the power of interactive computing to study modeling variations, results to confirm code correctness and describe performance, and a final showcase that computes solvation free energy for a medium-sized virus particle.
Our first result explains the behavior of two solution methods that vary in whether they solve for the potential internal or external to the molecular interface, from the conditioning point of view.
Solution verification via grid-convergence studies with two problem set-ups gives us confidence in the software implementation.
Performance-wise, we show results with problem sizes up to 2 million boundary elements, we show computational complexity of the \fmm evaluations, and timing breakdowns of the solver.
The final result uses a realistic structure, the enveloped Zika virus, computing the surface potential and solvation free energy with about 10 million boundary elements.
All results are reproducible and we share scripts, data, configuration files, and Jupyter notebooks in the manuscript repository, found at \url{https://github.com/barbagroup/bempp_exafmm_paper/}, in addition to permanent archives in Zenodo.
Permanent identifiers are provided at the end of the Results section.
